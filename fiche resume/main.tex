\documentclass[a4paper,french,12pt,reqno,dvipsnames,table]{memoir}

\usepackage{../utilities}

\usepackage{graphicx}
\usepackage{transparent} % Pour les trous
\usepackage{tabularx}                                               % Pour tableaux mais avec + de possibilités (tabular de base + mais tabularx offre possibilités ++)
\usepackage[para]{footmisc} % Pour notes de bas de pages sans retours à la lignes.
\usepackage{atbegshi} % Permet d'adjoidre un overlay tikz à chaque page (juste du contenu mais se précise en tikz de ma part)
\usepackage{lipsum} % Pour Lorem ipsum

\usetikzlibrary{backgrounds,positioning}
\usetikzlibrary{decorations.text}





\captionsetup[figure]{labelformat=empty} % Pour ne pas afficher le "Figure # : " dans les captions.



\definecolor{ListeClr1}{HTML}{17818e}
\newcommand{\pointListe}[1]{\textcolor{ListeClr1}{{\rmfamily \textbullet} \textbf{#1}}}

\definecolor{ListeClr2}{HTML}{0062ac}
\newenvironment{UneListe}
{
	\begin{itemize}[label=\textcolor{ListeClr2}{\textendash},noitemsep,topsep=0pt,parsep=0pt,partopsep=0pt,leftmargin=\labelwidth-2mm]
} {
	\end{itemize}
}

\def\quoteetc{[…]}

% Pour la police manuscrite par exemple pour citations de vieux textes historiques.
\newfontfamily\policemanuscrite{QTBlackForest}

\newfontfamily\policeTitreChapitre{QTAncientOlive}

\newfontfamily\afffont{QTLinostroke}
\newcommand{\polaff}[1]{\bgroup\afffont{#1}\egroup}

\newfontfamily\affnumfont{QTBengal}
\newcommand{\polnumaff}[1]{\bgroup\affnumfont{#1}\egroup}



% Paramétrage de la taille de la feuille.
\geometry{
	a4paper,
	nohead,
	total={170mm,257mm},
	left=20mm,
	right=20mm,
	top=20mm,
	bottom=25mm
}


\definecolor{clrsstxttirechap}{HTML}{b0b0b0}

\pagestyle{fancy}
\fancyhf{} % clear existing header/footer entries
\fancyfoot[C]{
	\thepage{} / \pageref*{LastPage}
}
\renewcommand{\headrulewidth}{0pt} % pour enlever la ligne mise en header de chaque page à cause de fancyhdr

\fancypagestyle{plain}{\fancyfoot[C]{\thepage{} / \pageref*{LastPage}}}

% Pour le padding des cellules dans l'env. tblr
\SetTblrInner{rowsep=2mm,colsep=2mm}



\tcbset{boite titre/.style={
		breakable,
		blanker,
		left=5mm,
		top=3mm,
		bottom=3mm,
		enhanced,
		colback=white
	}
}
\definecolor{couleurBordsTitresParties}{HTML}{0092ba}
\newtcolorbox{boitetitre}{boite titre,borderline west={1.5mm}{0pt}{couleurBordsTitresParties}}
\newcommand{\partie}[1]{
	\vspace{2mm}
	\begin{boitetitre}
		\textbf{\Large #1}
	\end{boitetitre}
	\vspace{1mm}
}

\definecolor{fondCitation}{HTML}{91bde3}
\tcbset{boite citation/.style={
		breakable,
		leftrule=3mm,
		left=3mm,
		top=3mm,
		bottom=3mm,
		enhanced,
		colback=fondCitation
	}
}
\newtcolorbox{folleCitation}{boite citation}


\definecolor{clraffnoshadow}{HTML}{cccccc}
% Sur chaque page
\AtBeginShipout {
	\AtBeginShipoutAddToBox{
		\begin{tikzpicture}[remember picture,overlay,shift=(current page.south west)]
			\node[rotate=90,anchor=south,clraffnoshadow] at ($(current page.east)+(0,-5)$) {\Huge \textcolor{clraffnoshadow}{\polaff{AFF} n°\polnumaff{17}}};
		\end{tikzpicture}
	}
}


\parindent=0mm % Pas d'indentation pour les paragraphes

\begin{document}




\begin{tikzpicture}[remember picture,overlay,shift=(current page.south west)]
	\fill[gray8] (current page.north west) rectangle ++(\pagewidth,-6);
\end{tikzpicture}

\definecolor{franceBleu}{HTML}{002654}
\definecolor{franceBlanc}{HTML}{ffffff}
\definecolor{franceRouge}{HTML}{ed2939}

\vspace*{-50pt}

\begin{figure}[H]
	\centering
	\begin{tikzpicture}
		\node[franceBleu]                    (A)      at (0,0)                 {\afffont \fontsize{70pt}{0pt}\selectfont \polaff{A}};
		\node[anchor=south west]             (sperge) at ($(A.south east)+(-0.2,0)$)   {\large sperge};
		\node[anchor=south west]             (de1)    at ($(sperge.north west)+(0.3,0)$)   {de};
		\node[anchor=south west,franceBlanc] (F1)     at ($(de1.south east)+(-0.2,0)$) {\fontsize{70pt}{0pt}\selectfont \polaff{F}};
		\node[anchor=south west]             (in)     at ($(F1.south east)+(-0.7,0)$)   {\large in};
		\node[anchor=south west]             (de2)    at (in.north east)   {de};
		\node[anchor=south west,franceRouge] (F2)     at ($(de2.south east)+(-0.2,0)$) {\fontsize{70pt}{0pt}\selectfont \polaff{F}};
		\node[anchor=south west]             (romagier)  at ($(F2.south east)+(-0.7,0)$)   {\large romagier};
		
		\node (dixsept) at ($(romagier.east)+(3,0)$) {\fontsize{100pt}{0pt}\selectfont \textbf{\polnumaff{17}}};
		\node[anchor=south east] (no) at ($(dixsept.south west)+(0.75,0.1)$) {\large n°};
		
	\end{tikzpicture}
\end{figure}

\vspace{30pt}

{\Large \underline{Catégorie}}

\vspace{-2pt}
\lipsum[1][3]

\vspace{4pt}
{\Large \underline{Description}}

\vspace{-2pt}
\lipsum[1][3]
\begin{itemize}
	\item ztieo tgn zepgnp zngioerngegnze nergnr ;
	\item zfgi zeni nei f p e i hrpfger gergerge ;
	\item zfgigjeri jnguer z cguhzeuibh geoh cez.
\end{itemize}


\def\proseNoA{Proust parlait de madeleine au beurre.}
\def\proseNoB{Kant a quant à lui parlé de madeleine à l'huile d'Argan.}

\vspace{30pt}

\definecolor{curve1}{HTML}{465892}
\definecolor{curve2}{HTML}{968eaf}
\definecolor{curve3}{HTML}{543d7b}
\tikz[overlay] {
	\draw[domain=-5:20,smooth,curve2] plot ({\x,0.01 * \x * \x - 0.1 * \x});
	\path[domain=-5:20,smooth,postaction={decoration={text along path, text={\proseNoA},text color=curve1,text align={align=center}}, decorate}] plot ({\x,0.01 * \x * \x - 0.1 * \x+0.1});
	\path[domain=-5:20,smooth,postaction={decoration={text along path, text={\proseNoB},text color=curve3,text align={align=center,left indent=2cm}}, decorate}] plot ({\x,0.01 * \x * \x - 0.1 * \x - 0.4});
}

\vspace{40pt}

\partie{Introduction}

\lipsum[1]

\partie{Deutroduction}

\lipsum[3]

\partie{Troitroduction}

\lipsum[1]

\begin{figure}[H]
	\centering
	\begin{tikzpicture}
		
		\node (calclit) at (0,0) {\large \textcolor{purple5}{\textbf{Item n°1}}};
		\node[anchor=north] (simpl1) at (calclit.south) {\textcolor{magenta5}{\textbf{\textendash}} Manger des fraises ;};
		\node[anchor=north west] (simpl2) at (simpl1.south west) {\textcolor{magenta5}{\textbf{\textendash}} Souffler de l'eau.};
		
		\node[anchor=north] (fracs) at ($(calclit.south)+(0,-2)$) {\large \textcolor{purple5}{\textbf{Item n°2}}};
		\node[anchor=north] (repr1) at (fracs.south) {\textcolor{magenta5}{\textbf{\textendash}} Fendre (un vase en terre cuite) ;};
		\node[anchor=north west] (repr2) at (repr1.south west) {\textcolor{magenta5}{\textbf{\textendash}} Supprimer une taxe ;};
		\node[anchor=north west] (repr3) at (repr2.south west) {\textcolor{magenta5}{\textbf{\textendash}} Marquer une marque.};
		
		\node[anchor=west] (grandsmes) at ($(simpl1.east)+(2.5,0)$) {\large \textcolor{purple5}{\textbf{Item n°3}}};
		\node[anchor=north] (grm1) at (grandsmes.south) {\textcolor{magenta5}{\textbf{\textendash}} Destruction de l'univers ;};
		\node[anchor=north west] (grm2) at (grm1.south west) {\textcolor{magenta5}{\textbf{\textendash}} Calculs $\alpha = \beta \div \gamma$ ;};
		\node[anchor=north west] (grm3) at (grm2.south west) {\textcolor{magenta5}{\textbf{\textendash}} Sommes de sons ;};
		\node[anchor=north west] (grm4) at (grm3.south west) {\textcolor{magenta5}{\textbf{\textendash}} Sommes de monts ;};
		\node[anchor=north west] (grm5) at (grm4.south west) {\textcolor{magenta5}{\textbf{\textendash}} BTP.};
		
	\end{tikzpicture}
\end{figure}

\lipsum[4]

\partie{Quatroduction}

\lipsum[1][4]

\begin{folleCitation}
	
	Une citation.
	
\end{folleCitation}

\definecolor{coul1}{HTML}{ac92eb}
\definecolor{coul2}{HTML}{4fc1e8}
\definecolor{coul3}{HTML}{a0d568}
\definecolor{coul4}{HTML}{ffce54}
\definecolor{coul5}{HTML}{ed5564}
\definecolor{coulmort}{HTML}{000000}

{
\tiny
\begin{figure}[H]
	\centering
	\begin{tblr}{
		colspec = {|c|X|},
		cell{1}{1} = {coul1},
		cell{2}{1} = {coul2},
		cell{3}{1} = {coul3},
		cell{4}{1} = {coul4},
		cell{5}{1} = {coul5},
		cell{6}{1} = {coulmort},
		hlines = {1pt},
		vlines = {1pt}
	}
		Velociraptor & Le vélociraptor était un petit dinosaure carnivore bipède qui vivait pendant le Crétacé supérieur. Il était connu pour sa vélocité, son intelligence et ses griffes redoutables, faisant de lui un prédateur redoutable dans son environnement. \\
		
		Pterosaur & Le ptérosaure était un reptile volant du Mésozoïque, caractérisé par ses ailes membranaires soutenues par un doigt allongé. Il était souvent confondu avec les dinosaures mais appartenait à un groupe distinct appelé Pterosauria. Ces créatures volantes variaient en taille, allant de la taille d'un petit oiseau à celle d'un avion léger. \\
		
		Dilophosaurus & Le dilophosaure était un dinosaure carnivore du Jurassique moyen, caractérisé par ses deux crêtes osseuses sur le crâne. De taille moyenne, il mesurait environ 6 mètres de longueur. Bien qu'il ait été popularisé par le film Jurassic Park avec une taille exagérée et la capacité à cracher du venin, ces caractéristiques ne sont pas soutenues par des preuves fossiles solides. \\
		
		Parasaurolophus & Le Parasaurolophus était un dinosaure herbivore du Crétacé supérieur, reconnaissable à sa grande crête creuse en forme de tube sur sa tête. Il mesurait environ 10 mètres de long et vivait en troupeaux dans les zones boisées et humides. \\
		
		Spinosaurus & Le Spinosaurus était un immense dinosaure carnivore aquatique du Crétacé, reconnaissable à sa voile osseuse sur le dos. Mesurant jusqu'à 15 mètres de long, il était un redoutable prédateur des milieux marins. \\
		
		\textcolor{white}{Michel} & \lipsum[1][5].
		
	\end{tblr}
\end{figure}
}

\begin{tikzpicture}
	\node (texte) at (0,0) {Niveau du dino le plus fort :};
	\node[anchor=west] at (texte.east) {\includegraphics[width=2cm]{niveau dino}};
\end{tikzpicture}

\begin{figure}[H]
	\centering
	\begin{tblr}{
		width=\linewidth,
		colspec = {|>{\centering\arraybackslash}X|>{\centering\arraybackslash}X|},
		hlines = {0pt},
		vlines = {0pt}
	}
		Niveau donné par Steve Jobs & Initiales et signature de Steve Jobs
	\end{tblr}
\end{figure}

\end{document}